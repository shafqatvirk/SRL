%
% File acl2013.tex
%
% Contact  navigli@di.uniroma1.it
%%
%% Based on the style files for ACL-2012, which were, in turn,
%% based on the style files for ACL-2011, which were, in turn, 
%% based on the style files for ACL-2010, which were, in turn, 
%% based on the style files for ACL-IJCNLP-2009, which were, in turn,
%% based on the style files for EACL-2009 and IJCNLP-2008...

%% Based on the style files for EACL 2006 by 
%%e.agirre@ehu.es or Sergi.Balari@uab.es
%% and that of ACL 08 by Joakim Nivre and Noah Smith

\documentclass[11pt]{article}
\usepackage{acl2013}
\usepackage{times}
\usepackage{url}
\usepackage{latexsym}
%\setlength\titlebox{6.5cm}    % You can expand the title box if you
% really have to

\title{A Multi-Classifier Based Semantic Role Labeling System for Chinese}

\author{First Author \\
  Affiliation / Address line 1 \\
  Affiliation / Address line 2 \\
  Affiliation / Address line 3 \\
  {\tt email@domain} \\\And
  Second Author \\
  Affiliation / Address line 1 \\
  Affiliation / Address line 2 \\
  Affiliation / Address line 3 \\
  {\tt email@domain}  \\\And
  Third Author \\
    Affiliation / Address line 1 \\
    Affiliation / Address line 2 \\
    Affiliation / Address line 3 \\
    {\tt email@domain} \\}

\date{}

\begin{document}
\maketitle
\begin{abstract} 
  The performance of a semantic role labeling (SRL) system depends on a number of factors including the feature set being used, syntactic representation of the sentence being labeled, and the underlying classification approach. Previously, a number of studies have been devoted to the selection of appropriate features, and to the effect of syntactic representation on semantic role labeling. In this paper, we study the effect of classification approach on SRL by comparing the performance of a number of systems based on different classification models (i.e. NaiveBays, Decision Trees, Maximum Entropy), and a system based on linear interpolation probability distribution model. We also propose a slightly different classification approach which is based on weighted simple probabilistic models. Experimental results show that, for the task of SRL, our approach is useful and outperforms those already existing classification approaches. Using our new approach, together with more general features to handle data sparseness, and a number of other heuristic rules, we report an improvement from 92.71\% to 94.8\% in the accuracy of Sinca parsers's semantic role labeling component.      
\end{abstract}

\section{Introduction}
Conventionally, almost all automatic semantic role labeling (SRL) systems take a pre-parsed sentence as input, extract a set of useful lexical and syntactical features from it, and then use a particular classification approach to label each semantic constituent of the sentence. Since semantic relationships are predictable from syntactic realizations \cite{Gildea:2002}, and the lexical features are valuable while predicting the constituent's semantic role, this makes the selection of an appropriate feature set very crucial. With \cite{Gildea:2002} being the pioneer, a number of other studies \cite{Xue04calibratingfeatures,Pradhan04shallowsemantic,sun-jurafsky:2004:HLTNAACL,Xue:2008:LCP:1403157.1403161} have highlighted the importance of different features, and tried to come-up with the best possible feature set. Similarly, the importance and influence of syntactic representation on SRL has been reported by a number of studies \cite{Johansson08theeffect,Gildea03identifyingsemantic,Pradhan:2008:TRS:1403157.1403163,Swanson06acomparison}. \\
However, very little work has been reported on the effect of chosen classification approach on SRL. Mostly,  the studies concentrate on the selection of useful features for SRL, on the syntactic representation, and use a particular classification technique in the classification phase of SRL. \cite{Gildea:2002}, however, have used equal linear interpolation, EM linear interpolation, geometric mean, and a backoff strategy to compare the performance of their system. Other than this, there is no detailed study on the effect of chosen classification approach on SRL.\\
In this study, we outline the influence of classification approach on SRL. We build a number of SRL systems, which are based on different classification approaches. In all systems, the input has the same syntactic representation and the same feature set in being used. First, they are trained on the same training data, then they are tested on the same testing data. Their performance is different from each other, owing much to the difference of their classification approach. Based on these findings, we propose a slightly different classification model, which is based on weighted simple probabilistic models. These models use different features and the probabilities are calculated from semantically labeled Sinical treebank \cite{ }. We then use a weighting strategy to rank their outputs, and then select the top ranked role to be assigned to a constituent. We use genetic algorithms for finding optimal weights. ...............\\
The rest of this paper is organized as: ....................


\section{Experiments}
\section{Evaluation}
\section{Discussion}
\section{Future Work}




%\begin{thebibliography}{}

%\bibitem[\protect\citename{Aho and Ullman}1972]{Aho:72}
%Alfred~V. Aho and Jeffrey~D. Ullman.
%\newblock 1972.
%\newblock {\em The Theory of Parsing, Translation and Compiling}, volume~1.
%\newblock Prentice-{Hall}, Englewood Cliffs, NJ.

%\bibitem[\protect\citename{{American Psychological Association}}1983]{APA:83}
%{American Psychological Association}.
%\newblock 1983.
%\newblock {\em Publications Manual}.
%\newblock American Psychological Association, Washington, DC.

%\bibitem[\protect\citename{{Association for Computing Machinery}}1983]{ACM:83}
%{Association for Computing Machinery}.
%\newblock 1983.
%\newblock {\em Computing Reviews}, 24(11):503--512.

%\bibitem[\protect\citename{Chandra \bgroup et al.\egroup }1981]{Chandra:81}
%Ashok~K. Chandra, Dexter~C. Kozen, and Larry~J. Stockmeyer.
%\newblock 1981.
%\newblock Alternation.
%\newblock {\em Journal of the Association for Computing Machinery},
 % 28(1):114--133.

%\bibitem[\protect\citename{Gusfield}1997]{Gusfield:97}
%Dan Gusfield.
%\newblock 1997.
%\newblock {\em Algorithms on Strings, Trees and Sequences}.
%\newblock Cambridge University Press, Cambridge, UK.

%\end{thebibliography}
\bibliographystyle{apalike}
\bibliography{srl-bib}
\end{document}
