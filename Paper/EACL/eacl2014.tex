%
% File eacl2014.tex
%
% Contact g.bouma@rug.nl yannick.parmentier@univ-orleans.fr
%
% Based on the instruction file for ACL 2013 
% which in turns was based on the instruction files for previous 
% ACL and EACL conferences

%% Based on the instruction file for EACL 2006 by Eneko Agirre and Sergi Balari
%% and that of ACL 2008 by Joakim Nivre and Noah Smith

\documentclass[11pt]{article}
\usepackage{eacl2014}
\usepackage{times}
\usepackage{url}
\usepackage{latexsym}
\special{papersize=210mm,297mm} % to avoid having to use "-t a4" with dvips 
%\setlength\titlebox{6.5cm}  % You can expand the title box if you really have to

\title{Instructions for EACL-2014 Proceedings}

\author{First Author \\
  Affiliation / Address line 1 \\
  Affiliation / Address line 2 \\
  Affiliation / Address line 3 \\
  {\tt email@domain} \\\And
  Second Author \\
  Affiliation / Address line 1 \\
  Affiliation / Address line 2 \\
  Affiliation / Address line 3 \\
  {\tt email@domain} \\}

\date{}

\begin{document}
\maketitle
\begin{abstract}
  This document contains the instructions for preparing a camera-ready
  manuscript for the proceedings of EACL-2014. The document itself
  conforms to its own specifications, and is therefore an example of
  what your manuscript should look like. These instructions should be
  used for both papers submitted for review and for final versions of
  accepted papers.  Authors are asked to conform to all the directions
  reported in this document.
\end{abstract}

\section{Credits}

This document has been adapted from the instructions for earlier
(E)ACL-proceedings, including those for ACL-2013 by Roberto Navigli
and Jing-Shing Chang, those from ACL-2012 by Maggie Li and Michael
White, those from ACL-2010 by Jing-Shing Chang and Philipp Koehn,
those for ACL-2008 by Johanna D. Moore, Simone Teufel, James Allan,
and Sadaoki Furui, those for EACL-2006 by Eneko Agirre and Sergi
Balari, those for ACL-2005 by Hwee Tou Ng and Kemal Oflazer, those for
ACL-2002 by Eugene Charniak and Dekang Lin, and earlier ACL and EACL
formats. Those versions were written by several people, including John
Chen, Henry S. Thompson and Donald Walker. Additional elements were
taken from the formatting instructions of the {\em International Joint
  Conference on Artificial Intelligence}.

\section{Introduction}
% das: removed reference to PostScript

The following instructions are directed to authors of papers submitted
to EACL-2014 or accepted for publication in its proceedings. All
authors are required to adhere to these specifications. Authors are
required to provide a Portable Document Format (PDF) version of their
papers. \textbf{The proceedings will be printed on A4 paper}. Authors
from countries in which access to word-processing systems is limited
should contact the publication chairs Gosse Bouma ({\small {\tt
  g.bouma@rug.nl}}) and Yannick Parmentier ({\small {\tt
  yannick.parmentier@univ-orleans.fr}}) as soon as possible.

\section{General Instructions}

Manuscripts must be in two-column format.  Exceptions to the
two-column format include the title, authors' names and complete
addresses, which must be centered at the top of the first page, and
any full-width figures or tables (see the guidelines in
Subsection~\ref{ssec:fonts}). {\bf Type single-spaced.}  Start all
pages directly under the top margin. See the guidelines later
regarding formatting the first page (Subsection~\ref{ssec:first}).
The manuscript should be printed single-sided and its length should
not exceed the maximum page limit described in
Section~\ref{sec:length}.  {\bf Do not number the pages.}

\subsection{Electronically-available resources}

EACL-2014 provides this description in \LaTeX2e ({\small {\tt
    eacl2014.tex}}) and PDF format ({\small {\tt eacl2014.pdf}}),
along with the \LaTeX2e style file used to format it ({\small {\tt
    eacl2014.sty}}) and an ACL bibliography style ({\small {\tt
    acl.bst}}) in case you want to use the Bib\TeX\ reference management
software. These files are all available at
\url{http://www.eacl2014.org}. A Microsoft Word template file ({\small
  {\tt eacl2014.docx}}) is also available at the same URL. We strongly
recommend the use of these style files, which have been appropriately
tailored for the EACL-2014 proceedings. If you have an option, we
recommend that you use the \LaTeX2e version. \textbf{If you will be
  using the Microsoft Word template, we suggest that you anonymize
  your source file so that the pdf produced does not retain your
  identity.}  This can be done by removing any personal information
from your source document properties.



\subsection{Format of Electronic Manuscript}
\label{sect:pdf}

For the production of the electronic manuscript you must use Adobe's
Portable Document Format (PDF). 
%% Add instructions about using pdflatex
This format can be generated directly from \LaTeX2e files or from
postscript ones. On Linux/Unix-like systems, you can use {\tt
  pdflatex} to generate a PDF file from \LaTeX2e files, or {\tt
  ps2pdf} to convert from postscript to PDF. 
%%
In Microsoft Windows, you can use Adobe's Distiller, or if you have
\texttt{cygwin} installed, you can use \texttt{dvipdf} or
\texttt{ps2pdf}. Note that some word processing programs generate PDF
which may not include all the necessary fonts (esp. tree diagrams,
symbols). When you print or create the PDF file, there is usually an
option in your printer setup to include none, all or just non-standard
fonts.  Please make sure that you select the option of including ALL
the fonts. {\em Before sending it, test your PDF by printing it from a
  computer different from the one where it was created.} Moreover,
some word processors may generate very large postscript/PDF files,
where each page is rendered as an image. Such images may reproduce
poorly. In this case, try alternative ways to obtain the postscript
and/or PDF. One way on some systems is to install a driver for a
postscript printer, send your document to the printer specifying
``Output to a file'', then convert the file to PDF.
Please keep in mind that it is of utmost importance to use \textbf{A4 format}.

%%Removed by YP (use of the \special command above instead):
%% It is of utmost importance to specify the \textbf{A4 format} (21 cm x
%% 29.7 cm) when formatting the paper. 
%% When working with {\tt dvips}, for instance, one should specify {\tt -t a4}.

Print-outs of the PDF file on A4 paper should be identical to the
hardcopy version. If you cannot meet the above requirements about the
production of your electronic submission, please contact the
publication chairs above as soon as possible.


\subsection{Layout}
\label{ssec:layout}

Format manuscripts two columns to a page, in the manner these
instructions are formatted. The exact dimensions for a page on A4
paper are:

\begin{itemize}
\item Left and right margins: 2.5 cm
\item Top margin: 2.5 cm
\item Bottom margin: 2.5 cm
\item Column width: 7.7 cm
\item Column height: 24.7 cm
\item Gap between columns: 0.6 cm
\end{itemize}

\noindent Papers should not be submitted on any other paper size.
 If you cannot meet the above requirements about the production of your electronic submission, 
please contact the publication chairs above as soon as possible.

\subsection{Fonts}
\label{ssec:fonts}
For reasons of uniformity, Adobe's {\bf Times Roman} font should be
used. In \LaTeX2e{} this is accomplished by putting

\begin{quote}
\begin{verbatim}
\usepackage{times}
\usepackage{latexsym}
\end{verbatim}
\end{quote}
in the preamble. If Times Roman is unavailable, use {\bf Computer
  Modern Roman} (\LaTeX2e{}'s default).  Note that the latter is about
  10\% less dense than Adobe's Times Roman font.


\begin{table}[h]
\begin{center}
\begin{tabular}{|l|rl|}
\hline \bf Type of Text & \bf Font Size & \bf Style \\ \hline
paper title & 15 pt & bold \\
author names & 12 pt & bold \\
author affiliation & 12 pt & \\
the word ``Abstract'' & 12 pt & bold \\
section titles & 12 pt & bold \\
document text & 11 pt  &\\
captions & 11 pt & \\
abstract text & 10 pt & \\
bibliography & 10 pt & \\
footnotes & 9 pt & \\
\hline
\end{tabular}
\end{center}
\caption{\label{font-table} Font guide. }
\end{table}

\subsection{The First Page}
\label{ssec:first}

Center the title, author's name(s) and affiliation(s) across both
columns. Do not use footnotes for affiliations. Do not include the
paper ID number assigned during the submission process. Use the
two-column format only when you begin the abstract.

{\bf Title}: Place the title centered at the top of the first page, in
a 15-point bold font. (For a complete guide to font sizes and styles, 
see Table~\ref{font-table}) Long titles should be typed on two lines without
a blank line intervening. Approximately, put the title at 2.5 cm from
the top of the page, followed by a blank line, then the author's
names(s), and the affiliation on the following line. Do not use only
initials for given names (middle initials are allowed). Do not format surnames
in all capitals (e.g., use ``Schlangen'' not ``SCHLANGEN'').
Do not format title and section headings in all capitals as well
except for proper names (such as ``BLEU'') that are conventionally
in all capitals.
The affiliation should contain the author's complete address, and if
possible, an electronic mail address. Leave about 2 cm between the
affiliation and the body of the first page.
The title, author names and addresses should be completely
identical to those entered to the electronical paper submission
website in order to maintain the consistency of author information
among all publications of the conference.

{\bf Abstract}: Type the abstract at the beginning of the first
column. The width of the abstract text should be smaller than the
width of the columns for the text in the body of the paper by about
0.6 cm on each side. Center the word {\bf Abstract} in a 12 point bold
font above the body of the abstract. The abstract should be a concise
summary of the general thesis and conclusions of the paper. It should
be no longer than 200 words. The abstract text should be in 10 point font.

{\bf Text}: Begin typing the main body of the text immediately after
the abstract, observing the two-column format as shown in 
the present document. Do not include page numbers.

{\bf Indent} when starting a new paragraph. Use 11 points for text and 
subsection headings, 12 points for section headings and 15 points for
the title. 

\subsection{Sections}

{\bf Headings}: Type and label section and subsection headings in the
style shown on the present document.  Use numbered sections (Arabic
numerals) in order to facilitate cross references. Number subsections
with the section number and the subsection number separated by a dot,
in Arabic numerals. Do not number subsubsections.

{\bf Citations}: Citations within the text appear
in parentheses as~\cite{Gusfield:97} or, if the author's name appears in
the text itself, as Gusfield~\shortcite{Gusfield:97}. 
Append lowercase letters to the year in cases of ambiguity.  
Treat double authors as in~\cite{Aho:72}, but write as
 in~\cite{Chandra:81} when more than two authors are involved. Collapse multiple citations as
in~\cite{Gusfield:97,Aho:72}. Also refrain from using full citations as sentence constituents. We
suggest that instead of
\begin{quote}
  ``\cite{Gusfield:97} showed that ...''
\end{quote}
you use
\begin{quote}
``Gusfield \shortcite{Gusfield:97}   showed that ...''
\end{quote}

If you are using the provided \LaTeX{} and Bib\TeX{} style files, you
can use the command \verb|\newcite| to get ``author (year)'' citations.

As reviewing will be double-blind, the submitted version of the papers should not include the
authors' names and affiliations. Furthermore, self-references that
reveal the author's identity, e.g.,
\begin{quote}
``We previously showed \cite{Gusfield:97} ...''  
\end{quote}
should be avoided. Instead, use citations such as 
\begin{quote}
``Gusfield \shortcite{Gusfield:97}
previously showed ... ''
\end{quote}

\textbf{Please do not  use anonymous citations} and  do not include acknowledgements 
when submitting your papers. Papers that do not conform
to these requirements may be rejected without review. 

\textbf{References}: Gather the full set of references together under
the heading {\bf References}; place the section before any Appendices,
unless they contain references. Arrange the references alphabetically
by first author, rather than by order of occurrence in the text.
Provide as complete a citation as possible, using a consistent format,
such as the one for {\em Computational Linguistics\/} or the one in the 
{\em Publication Manual of the American 
Psychological Association\/}~\cite{APA:83}.  Use of full names for
authors rather than initials is preferred.  A list of abbreviations
for common computer science journals can be found in the ACM 
{\em Computing Reviews\/}~\cite{ACM:83}.

The \LaTeX{} and Bib\TeX{} style files provided roughly fit the
American Psychological Association format, allowing regular citations, 
short citations and multiple citations as described above.

{\bf Appendices}: Appendices, if any, directly follow the text and the
references (but see above).  Letter them in sequence and provide an
informative title: {\bf Appendix A. Title of Appendix}.

\textbf{Acknowledgement} section should appear in accepted manuscripts only. It should go as a last section immediately
before the references.  Do not number the acknowledgement section.

\subsection{Footnotes}

{\bf Footnotes}: Put footnotes at the bottom of the page and use 9
points text. They may be numbered or referred to by asterisks or other
symbols.\footnote{This is how a footnote should appear.} Footnotes
should be separated from the text by a line.\footnote{Note the line
separating the footnotes from the text.}

\subsection{Graphics}

{\bf Illustrations}: Place figures, tables, and photographs in the
paper near where they are first discussed, rather than at the end, if
possible.  Wide illustrations may run across both columns.  Color
illustrations are allowed, provided you have verified that  
they will be understandable when printed in black ink.

{\bf Captions}: Provide a caption for every illustration; number each one
sequentially in the form:  ``Figure 1. Caption of the Figure.'' ``Table 1.
Caption of the Table.''  Type the captions of the figures and 
tables below the body, using 11 point text.  

\section{Translation of non-English Terms}

It is also advised to supplement non-English characters and terms
with appropriate transliterations and/or translations
since not all readers understand all such characters and terms.
Inline transliteration or translation can be represented in
the order of: original-form transliteration ``translation''.

\section{Length of Submission}
\label{sec:length}

Long papers may consist of up to 8 pages of content (excluding
references) and short papers may consist of up to 4 pages (including
references) in the proceedings.  Papers that do not conform to the
specified length and formatting requirements are subject to be
rejected without review.

\section{Other Issues}

Papers that had software and/or dataset submitted for the review
process should also submit it with the camera-ready paper. Besides,
the software and/or dataset should not be anonymous.

Please note that the publications of EACL-2014 will be publicly
available at ACL Anthology (http://aclweb.org/anthology-new/) on April
19th, 2014, one week before the start of the conference. Since some of
the authors may have plans to file patents related to their papers in
the conference, we are reminding authors that April 19th, 2014 may be
considered to be the official publication date, instead of the opening
day of the conference.

\section*{Acknowledgments}

Do not number the acknowledgment section. Do not include this section
when submitting your paper for review.

% If you use BibTeX with a bib file named eacl2014.bib, 
% you should add the following two lines:
\bibliographystyle{acl}
\bibliography{eacl2014}

% Otherwise you can include your references as follows:
%% \begin{thebibliography}{}

%% \bibitem[\protect\citename{Aho and Ullman}1972]{Aho:72}
%% Alfred~V. Aho and Jeffrey~D. Ullman.
%% \newblock 1972.
%% \newblock {\em The Theory of Parsing, Translation and Compiling}, volume~1.
%% \newblock Prentice-{Hall}, Englewood Cliffs, NJ.

%% \bibitem[\protect\citename{{American Psychological Association}}1983]{APA:83}
%% {American Psychological Association}.
%% \newblock 1983.
%% \newblock {\em Publications Manual}.
%% \newblock American Psychological Association, Washington, DC.

%% \bibitem[\protect\citename{{Association for Computing Machinery}}1983]{ACM:83}
%% {Association for Computing Machinery}.
%% \newblock 1983.
%% \newblock {\em Computing Reviews}, 24(11):503--512.

%% \bibitem[\protect\citename{Chandra \bgroup et al.\egroup }1981]{Chandra:81}
%% Ashok~K. Chandra, Dexter~C. Kozen, and Larry~J. Stockmeyer.
%% \newblock 1981.
%% \newblock Alternation.
%% \newblock {\em Journal of the Association for Computing Machinery},
%%   28(1):114--133.

%% \bibitem[\protect\citename{Gusfield}1997]{Gusfield:97}
%% Dan Gusfield.
%% \newblock 1997.
%% \newblock {\em Algorithms on Strings, Trees and Sequences}.
%% \newblock Cambridge University Press, Cambridge, UK.

%% \end{thebibliography}

\end{document}
